% ------------------------------------------------------------------------------
% Centro Federal de Educação Tecnológica de Minas Gerais - CEFET-MG
%
% Modelo de trabalho acadêmico em conformidade com as normas da ABNT
% (Tese de Doutorado, Dissertação de Mestrado ou Projeto de Qualificação)
%
% Projeto hospedado em: https://github.com/cfgnunes/latex-cefetmg
%
% Autores: Cristiano Nunes <cfgnunes@gmail.com>
%          Henrique Borges <henrique@cefetmg.br>
% ------------------------------------------------------------------------------

\documentclass[%
    %twoside,              % Impressão em frente (anverso) e verso
    oneside,               % Impressão apenas no anverso
]{cefetmg}

\usepackage[%
    alf,
    abnt-emphasize=bf,
    bibjustif,
    recuo=0cm,
    abnt-doi=expand,
    abnt-url-package=hyperref,
    abnt-refinfo=yes,
    abnt-etal-cite=3,
    abnt-etal-list=3,
    abnt-thesis-year=final
]{abntex2cite}             % Configura as citações bibliográficas

% ------------------------------------------------------------------------------
% Pacotes utilizados
% ------------------------------------------------------------------------------
\usepackage[utf8]{inputenc}         % Codificação do documento
\usepackage[T1]{fontenc}            % Seleção de código de fonte
\usepackage{booktabs}               % Réguas horizontais em tabelas
\usepackage{color, colortbl}        % Controle de cores
\usepackage{float}                  % Tabelas/figuras em ambiente multicolunas
\usepackage{graphicx}               % Inclusão de gráficos
\usepackage{hyperref}               % Utilização de hyperlinks
\usepackage{icomma}                 % Uso de vírgulas em expressões matemáticas
\usepackage{indentfirst}            % Recua o primeiro parágrafo de cada seção
\usepackage{microtype}              % Melhora a justificação do documento
\usepackage{multirow, makecell}     % Tabelas com múltiplas linhas e colunas
\usepackage{subeqnarray}            % Subenumeração de equações
\usepackage{verbatim}               % Exibir texto tal como escrito no documento
\usepackage[algoruled, portuguese]{algorithm2e}  % Algoritmos em português
\usepackage{amsmath}                % Fontes e símbolos matemáticos
\usepackage[charter]{mathdesign}    % Usa a fonte Charter BT
%\usepackage{times}                 % Usa a fonte Times
%\usepackage{newtxtext,newtxmath}   % Usa a fonte Times (melhorado)
%\usepackage{palatino}              % Usa a fonte Palatino
%\usepackage{lmodern}               % Usa a fonte Latin Modern
%\usepackage{latexsym}              % Símbolos matemáticos
%\usepackage{lscape}                % Páginas em modo paisagem

% Utiliza uma fonte similar a Arial (fonte Helvetica)
%\usepackage{helvet}
%\renewcommand*\familydefault{\sfdefault}

% ------------------------------------------------------------------------------
% Configurações de aparência do PDF final
% ------------------------------------------------------------------------------
\makeatletter
\hypersetup{%
    portuguese,
    colorlinks=true, % true: links coloridos; false: links em caixas
    linkcolor=blue,  % Cor dos links internos
    citecolor=blue,  % Cor dos links para as referências bibliográficas
    filecolor=blue,  % Cor dos links para arquivos
    urlcolor=blue,   % Cor dos links de URLs
    breaklinks=true,
    pdftitle={\@title},
    pdfauthor={\@author},
    pdfkeywords={abnt, latex, abntex, abntex2}
}
\makeatother

% Altera a tonalidade da cor azul dos links
\definecolor{blue}{RGB}{0,80,128}

% Redefinição de labels
\renewcommand{\algorithmautorefname}{Algoritmo}

% Cria o índice remissivo
\makeindex

% Hifenização de palavras que não estão no dicionário
\hyphenation{%
    qua-dros-cha-ve
    Kat-sa-gge-los
}

% ------------------------------------------------------------------------------
% Inclui os arquivos do trabalho acadêmico
% ------------------------------------------------------------------------------

% Insere alguns elementos pré-textuais para gerar capa,
% folha de rosto e folha de aprovação
% -----------------------------------------------------------------------------
% Capa
% -----------------------------------------------------------------------------


% -----------------------------------------------------------------------------
% ATENÇÃO:
% Caso algum campo não se aplique ao seu documento - por exemplo, em seu trabalho
% não houve coorientador - não comente o campo, apenas deixe vazio, assim: \campo{}
% -----------------------------------------------------------------------------


% -----------------------------------------------------------------------------
% Dados do trabalho acadêmico
% -----------------------------------------------------------------------------

\titulo{Título do Trabalho}
%\title{Title in English}
\subtitulo{Subtítulo do trabalho}
\autor{Nome completo do autor}
\local{Belo Horizonte}
\data{Junho de 2016}            % Normalmente se usa apenas mês e ano


% -----------------------------------------------------------------------------
% Natureza do trabalho acadêmico
% Use apenas uma das opções: Tese (p/ Doutorado), Dissertação (p/ Mestrado) ou
% Projeto de Qualificação (p/ Mestrado ou Doutorado), Trabalho de Conclusão de
% Curso (Graduação)
% -----------------------------------------------------------------------------

\projeto{Projeto de Qualificação}


% -----------------------------------------------------------------------------
% Título acadêmico
% Use apenas uma das opções:
%   Se a natureza for Tese, coloque Doutor
%   Se a natureza for Dissertação, coloque Mestre
%   Se a natureza for Projeto de Qualificação, coloque Mestre ou Doutor conforme o caso
% Se a natureza for Trabalho de Conclusão de Curso, coloque Bacharel
% -----------------------------------------------------------------------------

\tituloAcademico{Doutor}


% -----------------------------------------------------------------------------
% Área de concentração e linha de pesquisa
%   OBS: indique o nome da área de concentração e da linha de pesquisa do Programa de Pós-graduação
% nas quais este trabalho se insere
% Se a natureza for Trabalho de Conclusão de Curso, deixe ambos os campos vazios
% -----------------------------------------------------------------------------

\areaconcentracao{Modelagem Matemática e Computacional}
\linhapesquisa{Sistemas Inteligentes}


% -----------------------------------------------------------------------------
% Dados da instituição
% OBS: a logomarca da instituição deve ser colocada na mesma pasta que foi colocada o documento
% principal com o nome de "logoInstituicao". O formato pode ser: pdf, jpf, eps
% Se a natureza for Trabalho de Conclusão de Curso, coloque em "programa' o nome do curso de graduação
% -----------------------------------------------------------------------------

\instituicao{Centro Federal de Educação Tecnológica de Minas Gerais}
\programa{Programa de Pós-graduação em Modelagem Matemática e Computacional}
%\programa{Curso de Engenharia de Computação}
\logoinstituicao{0.2}{./04-figuras/logo-instituicao}                  % \logoinstituicao{<escala>}{<nome do arquivo>}


% -----------------------------------------------------------------------------
% Dados do(s) orientador(es)
% -----------------------------------------------------------------------------

\orientador{Nome do orientador}
%\orientador[Orientadora:]{Nome da orientadora}
\instOrientador{Instituição do orientador}

\coorientador{Nome do coorientador}
%\coorientador[Coorientadora:]{Nome da coorientadora}
\instCoorientador{Instituição do coorientador}

% -----------------------------------------------------------------------------
% Folha de Rosto
% -----------------------------------------------------------------------------

% Trabalho de Conclusão de Curso
%\preambulo{{\imprimirprojeto} apresentado ao Curso de Engenharia de Computação do Centro Federal de Educação Tecnológica de Minas Gerais, como requisito parcial para a obtenção do título de {\imprimirtituloAcademico} em Engenharia de Computação.}

% Projeto de qualificação de Mestrado ou Doutorado
\preambulo{{\imprimirprojeto} apresentado ao Programa de \mbox{Pós-graduação} em Modelagem Matemática e Computacional do Centro Federal de Educação Tecnológica de Minas Gerais, como requisito parcial para a obtenção do título de {\imprimirtituloAcademico} em Modelagem Matemática e Computacional.}

% Dissertação de Mestrado
%\preambulo{{\imprimirprojeto} apresentada ao Programa de \mbox{Pós-graduação} em Modelagem Matemática e Computacional do Centro Federal de Educação Tecnológica de Minas Gerais, como requisito parcial para a obtenção do título de {\imprimirtituloAcademico} em Modelagem Matemática e Computacional.}

% Tese de Doutorado
%\preambulo{{\imprimirprojeto} apresentada ao Programa de \mbox{Pós-graduação} em Modelagem Matemática e Computacional do Centro Federal de Educação Tecnológica de Minas Gerais, como requisito parcial para a obtenção do título de {\imprimirtituloAcademico} em Modelagem Matemática e Computacional.}

% -----------------------------------------------------------------------------
% Edite este arquivo comentando as linhas que não se aplicam ao tipo de
% documento acadêmico pretendido.
% -----------------------------------------------------------------------------

% -----------------------------------------------------------------------------
% Folha de Aprovação
% -----------------------------------------------------------------------------

\textopadraofolhadeaprovacao{Esta folha deverá ser substituída pela cópia digitalizada da folha de aprovação fornecida.}

% -----------------------------------------------------------------------------
% Este documento foi mantido apenas para preservar a paginação do trabalho
% acadêmico final, após a inserção da folha de aprovação fornecida
% -----------------------------------------------------------------------------


\begin{document}

% Configura layout para os elementos pré-textuais
\pretextual
\imprimircapa                                       % Capa
\imprimirfolhaderosto{}                             % Folha de rosto
\imprimirfolhadeaprovacao{}                         % Folha de aprovação
%
% Documento: Dedicatória
%

\begin{dedicatoria}

Espaço reservado para dedicatória.
Inserir seu texto aqui...

\end{dedicatoria}
          % Dedicatória
% -----------------------------------------------------------------------------
% Agradecimentos
% -----------------------------------------------------------------------------

\begin{agradecimentos}

    Edite e coloque aqui os agradecimentos às pessoas e/ou instituições que contribuíram para a realização do trabalho.

    É obrigatório o agradecimento às instituições de fomento à pesquisa que financiaram total ou parcialmente o trabalho, inclusive no que diz respeito à concessão de bolsas.

\end{agradecimentos}
       % Agradecimentos
%
% Documento: Epígrafe
%

\begin{epigrafe}
    \vspace*{\fill}
	\begin{flushright}
		\textit{``O fator decisivo para vencer o maior obstáculo é, invariavelmente, ultrapassar o obstáculo anterior.'' (Henry Ford)}
	\end{flushright}
\end{epigrafe}
             % Epígrafe
\input{elementos-pre-textuais/resumo-pt}            % Resumo
% -----------------------------------------------------------------------------
% Abstract
% -----------------------------------------------------------------------------

\begin{resumo}[Abstract]

    Translation of the abstract into english, possibly adapting or slightly changing the text in order to adjust it to the grammar of Standard English.
    Try to stay within the limit of: 500 word for a PhD Thesis;
    250 words for a Master Dissertation;
    200 words for a Qualifying Research Project.

    \par\vspace{\baselineskip}

    \textbf{Keywords}: Latex model. Academic work. ABNT standards. Another word.
\end{resumo}

% Em uma Tese de Doutorado o resumo deve conter, no máximo, 500 palavras.
% Em uma Dissertação de Mestrado o resumo deve conter, no máximo, 250 palavras.
% Em um Projeto de Qualificação o resumo deve conter, no máximo, 200 palavras.

% -----------------------------------------------------------------------------
% O restante da formatação deve manter-se igual ao do resumo em português,
% por exemplo, um único parágrafo.
% -----------------------------------------------------------------------------
            % Abstract
\imprimirlistafiguras                               % Lista de figuras
\imprimirlistatabelas                               % Lista de tabelas
\imprimirlistaquadros                               % Lista de quadros
\imprimirlistaalgoritmos                            % Lista de algoritmos
% -----------------------------------------------------------------------------
% Lista de Siglas
% -----------------------------------------------------------------------------

\begin{siglas}
    \item[ABNT] Associação Brasileira de Normas Técnicas
    \item[DECOM] Departamento de Computação
\end{siglas}

% -----------------------------------------------------------------------------
% Edite a lista acima para definir "todos" os acrônimos e siglas utilizados neste trabalho
% -----------------------------------------------------------------------------
         % Lista de siglas
% -----------------------------------------------------------------------------
% Lista de Símbolos
% -----------------------------------------------------------------------------

\begin{simbolos}
    \item[$ \Gamma $] Letra grega Gama
    \item[$ \lambda $] Comprimento de onda
    \item[$ \in $] Pertence
\end{simbolos}

% -----------------------------------------------------------------------------
% Edite a lista acima para definir os símbolos utilizados neste trabalho.
% -----------------------------------------------------------------------------
       % Lista de símbolos
\imprimirsumario                                    % Sumário

% Configura layout para os elementos textuais
\textual
% ------------------------------------------------------------------------------
% Introdução
% ------------------------------------------------------------------------------

\chapter{Introdução}
\label{chap_introducao}

A introdução deverá apresentar uma visão de conjunto do trabalho a ser realizado, com o apoio da literatura, situando-o no contexto do estado da arte da área científica específica, sua relevância no contexto da área inserida e sua importância específica para o avanço do conhecimento.

É uma boa prática iniciar cada novo capítulo com um breve texto introdutório (tipicamente, dois ou três parágrafos) que deve deixar claro o quê será discutido no capítulo, bem como a organização do mesmo.
Também servirá ao propósito de ``amarrar'' ou ``alinhavar'' o conteúdo deste capítulo com o conteúdo do capítulo imediatamente anterior.

\section{Motivação}
\label{sec_motivacao}

Este documento é um \emph{template} que foi concebido, primariamente, para ser utilizado na redação de teses de doutorado, dissertações de mestrado, projetos de qualificação tanto de mestrado quanto de doutorado, escritos em português brasileiro (eventualmente, com partes em inglês) e em conformidade com as normas da ABNT.

Não obstante, ele também poderá ser utilizado, com ligeiras adaptações para a redação de outros trabalhos acadêmicos monográficos (e.g., trabalhos de conclusão de curso de graduação ou de especialização \emph{lato sensu}).

Antes de começar a escrever o seu trabalho acadêmico utilizando este \emph{template}, é bom saber que há um arquivo que você precisará editar para que a capa e a folha de rosto de seu trabalho sejam geradas.
Este arquivo é o {\color{red} preambulo.tex} e se encontra no diretório {\color{red} elementos-pre-textuais}.
Nesse arquivo, você deverá informar o seu nome, título do trabalho acadêmico, se o documento será uma tese de doutorado ou dissertação de mestrado ou projeto de qualificação, nome de seu(s) orientador(es), e outras informações necessárias.

Para compilar o documento, você pode utilizar o arquivo {\color{red} makefile} por meio do comando {\color{red} make}, disponível na mesma pasta onde está o arquivo principal {\color{red} meu-\allowbreak trabalho.tex}.
No entanto, atente para o fato de que, se você alterar o nome do arquivo {\color{red} meu-\allowbreak trabalho.tex}, deverá também editar o arquivo {\color{red} makefile} para alterá-lo do mesmo modo.

Por fim, caso observe algum problema ou qualquer outro tipo de falha ou mal comportamento neste modelo, comunique-nos para que possamos tentar corrigi-los em futuras atualizações.

\section{Definição do problema de pesquisa}
\label{sec_definicao_problema_pesquisa}

Inserir seu texto aqui...

\section{Objetivos}
\label{sec_objetivos}

Inserir seu texto aqui...

\section{Contribuições}
\label{sec_contribuicoes}

Inserir seu texto aqui...

\section{Organização do trabalho}
\label{sec_organizacao_trabalho}

Normalmente ao final da introdução é apresentada, em um ou dois parágrafos curtos, a organização do restante do trabalho acadêmico.
Deve-se dizer o quê será apresentado em cada um dos demais capítulos.

Este trabalho está organizado em capítulos, incluindo o presente.
No \autoref{chap_fundamentacao_teorica} são apresentados alguns dos principais conceitos necessários que fundamentam o desenvolvimento deste trabalho.
A \hyperref[chap_trabalhos_relacionados]{revisão bibliográfica} deste trabalho apresenta uma revisão dos principais estudos relacionados ao tema, descrevendo seus resultados e suas contribuições.
Por fim, no \autoref{chap_conclusao} são apresentadas as conclusões, bem como as perspectivas de trabalhos futuros.
               % Introdução
% -----------------------------------------------------------------------------
% Trabalhos Relacionados
% -----------------------------------------------------------------------------

\chapter{Trabalhos Relacionados}
\label{chap:trabalhos_relacionados}

Cada capítulo deve conter uma pequena introdução (tipicamente, um ou dois parágrafos), em seção não numerada, que deve deixar claro o objetivo e o que será discutido no capítulo, bem como a organização do capítulo.
   % Trabalhos relacionados
% ------------------------------------------------------------------------------
% Fundamentação Teórica
% ------------------------------------------------------------------------------

\chapter{Fundamentação Teórica}
\label{chap_fundamentacao_teorica}

É uma boa prática iniciar cada novo capítulo com uma breve texto introdutório (tipicamente, dois ou três parágrafos) que deve deixar claro o quê será discutido no capítulo, bem como a organização do capítulo.
Também servirá ao propósito de ``amarrar'' ou ``alinhavar'' o conteúdo deste capítulo com o conteúdo do capítulo imediatamente anterior.
    % Fundamentação teórica
% ------------------------------------------------------------------------------
% Metodologia
% ------------------------------------------------------------------------------

\chapter{Metodologia}
\label{chap:metodologia}
Cada capítulo deve conter uma pequena introdução (tipicamente, um ou dois parágrafos), em seção não numerada, que deve deixar claro o objetivo e o que será discutido no capítulo, bem como a organização do capítulo.

\section{Delineamento da pesquisa}
\label{sec:delineamento_da_pesquisa}

Inserir seu texto aqui...

\section{Coleta e tratamento de dados}
\label{sec:coleta_e_tratamento_de_dados}

Inserir seu texto aqui...
              % Metodologia
% ------------------------------------------------------------------------------
% Resultados
% ------------------------------------------------------------------------------

\chapter{Análise e Discussão dos Resultados}

Cada capítulo deve conter uma pequena introdução (tipicamente, um ou dois parágrafos), que deve deixar claro o objetivo e o que será discutido no capítulo, bem como a organização do mesmo.

\section{Título da seção}
\label{sec_titulo_da_secao_resultados}

Inserir seu texto aqui...
               % Resultados
% ------------------------------------------------------------------------------
% Conclusão
% ------------------------------------------------------------------------------

\chapter{Conclusão}
\label{chap_conclusao}

Procure fazer uma análise crítica de seu trabalho, destacando os principais resultados e as contribuições deste trabalho para a área de pesquisa.

\section{Trabalhos futuros}
\label{sec_trabalhos_futuros}

Também deve indicar, se possível e/ou conveniente, como este trabalho pode ser estendido ou aprimorado.

% ------------------------------------------------------------------------------
% Observação: A norma ABNT estabelece que, em qualquer categoria de trabalho
% acadêmico monográfico deve haver um capítulo de conclusão
% ------------------------------------------------------------------------------
                % Conclusão

% Configura layout para os elementos pós-textuais
\postextual
\imprimirreferencias                                % Referências
% -----------------------------------------------------------------------------
% Apêndices
% -----------------------------------------------------------------------------

\begin{apendicesenv}
\partapendices

% -----------------------------------------------------------------------------
% Primeiro apêndice
% -----------------------------------------------------------------------------

\chapter{Nome do apêndice}
\label{chap:apendice_a}

Lembre-se que a diferença entre apêndice e anexo diz respeito à autoria do texto e/ou material ali colocado.

Caso o material ou texto suplementar ou complementar seja de sua autoria, então ele deverá ser colocado como um apêndice. Porém, caso a autoria seja de terceiros, então o material ou texto deverá ser colocado como anexo.

Caso seja conveniente, podem ser criados outros apêndices para o seu trabalho acadêmico. Basta recortar e colar este trecho neste mesmo documento. Lembre-se de alterar o ``label'' do apêndice.

% -----------------------------------------------------------------------------
% Novo apêndice
% -----------------------------------------------------------------------------

\chapter{Estrutura de trabalhos acadêmicos}
\label{chap:estrutura_de_trabalhos_academicos}

Quanto à estrutura do trabalho acadêmico, esta varia sobremaneira, a depender da conveniência do autor e seu(s) respectivo(s) orientador(es). No entanto, de acordo com as normas ABNT, alguns elementos são obrigatórios.

A título de sugestão, e apenas isso, a \autoref{fig:estrutura_projeto_qualificacao} apresenta uma estrutura para um projeto de qualificação de mestrado ou doutorado, conforme a norma \citeonline{NBR14724:2011}.

\begin{figure}[!htb]
    \centering
    \caption{Estrutura sugerida de um Projeto de Qualificação para os cursos de Mestrado ou Doutorado}
    \includegraphics[width=0.5\textwidth]{./figuras/estrutura-projeto-qualificacao}
    \label{fig:estrutura_projeto_qualificacao}
\end{figure}

Já a \autoref{fig:estrutura_tese_dissertacao} apresenta uma estrutura para uma tese de doutorado ou dissertação de mestrado, conforme a norma \citeonline{NBR14724:2011}.

Cabe ressaltar que, em todas as figuras, os elementos obrigatórios estão destacados em vermelho, os demais são opcionais.

\begin{figure}[!htb]
    \centering
    \caption{Estrutura sugerida de uma Tese de Doutorado ou Dissertação de Mestrado}
    \includegraphics[width=0.5\textwidth]{./figuras/estrutura-tese-dissertacao}
    \label{fig:estrutura_tese_dissertacao}
\end{figure}

Observe que a estrutura de um projeto de qualificação é muito similar à da tese ou dissertação. A única diferença existente é que num projeto de qualificação o autor certamente terá, via de regra, apenas resultados parciais e preliminares. Além disso, estando o trabalho ainda em andamento, há que se apresentar um cronograma de trabalho que evidencie que o mesmo poderá ser concluído dentro dos prazos estabelecidos pelo programa.

Por fim, como foi dito, este  \emph{template} pode ser utilizado para outros trabalhos acadêmicos. Neste caso, a \autoref{fig:estrutura_projeto_pesquisa} apresenta uma sugestão de projeto de pesquisa a ser submetido ao programa para fins de admissão ao mesmo, conforme a norma \citeonline{NBR15287:2005}.

\begin{figure}[!htb]
    \centering
    \caption{Estrutura sugerida de um projeto de pesquisa para admissão ao PPGMMC}
    \includegraphics[width=0.6\textwidth]{./figuras/estrutura-projeto-pesquisa}
    \label{fig:estrutura_projeto_pesquisa}
\end{figure}

Você deverá editar o arquivo principal {\ttfamily meu-trabalho.tex} para fazer os ajustes necessários, reiterando que as estruturas apresentadas são mera sugestão.

A inclusão de reticências (\ldots) no texto deverá ser feita através de um comando especial denominado \verb|\ldots| \cite{LaTeX2014}. Assim esse comando deverá ser utilizado ao invés da digitação de três pontos.

Para melhor entendimento do uso do estilo de formatação, aconselha-se que o potencial usuário analise os comandos existentes no arquivo {\ttfamily meu-trabalho.tex} e os resultados obtidos no arquivo {\ttfamily meu-trabalho.pdf}.
Recomenda-se a consulta ao material de referência do software para a sua correta utilização \cite{Lamport1986,Buerger1989,Kopka2003,Mittelbach2004}.

Finalmente, este modelo apresenta um arquivo {\ttfamily makefile} para agilizar a compilação do documento. Portanto, para gerar o documento final no formato PDF, basta apenas executar o comando {\ttfamily make} no linux. Para limpar os arquivos temporários, basta digitar o comando {\ttfamily make clean}.

% -----------------------------------------------------------------------------
% Novo apêndice
% -----------------------------------------------------------------------------

\chapter{Sobre as ilustrações}
\label{chap:sobre_as_ilustracoes}

A seguir ilustra-se a forma de incluir ilustrações no corpo do texto. Pela norma figuras, tabelas, quadros, equações, quadros, algoritmos, diagrama, etc. são tipos específicos de ilustrações. As ilustrações (pelo menos alguns tipos específicos) serão indexadas automática em suas respectivas listas.

A numeração sequencial de figuras, tabelas e equações ocorre de modo automático.

Referências cruzadas são obtidas através dos comandos \verb|\label{}| e \verb|\ref{}|. Por exemplo, não é necessário saber que o número de certo capítulo é \ref{chap:fundamentacao_teorica} para colocar o seu número no texto. Alternativamente se pode usar desta forma: \autoref{chap:fundamentacao_teorica}. Isto facilita muito a inserção, remoção ou relocação de elementos numerados no texto (fato corriqueiro na escrita e correção de um documento acadêmico) sem a necessidade de renumerá-los todos.

\section{Figuras}
\label{sec:figuras}

A seguir é apresentado um exemplo de figura.
A \autoref{fig:figura_exemplo} aparece automaticamente na lista de figuras.

\begin{figure}[!htb]
    \centering
    \caption{Exemplo de figura}
    \includegraphics[width=0.6\textwidth]{./figuras/figura-exemplo}
    \label{fig:figura_exemplo}
\end{figure}

\section{Quadros e tabelas}
\label{sec:tabelas}

Também é apresentado o exemplo do \autoref{qua:comparabd} e da \autoref{tab:testes}, que aparece automaticamente na lista de quadros e tabelas.

Informações sobre a construção de tabelas no \LaTeX{} podem ser encontradas na literatura especializada \cite{Lamport1986,Buerger1989,Kopka2003,Mittelbach2004}.

\begin{quadro}[!htb]
    \centering
    \caption{Hierarquia de restrições das questões.}
    \begin{tabular}{|p{7cm}|p{7cm}|}
        \hline
        \textbf{BD Relacionais}                                                                       & \textbf{BD Orientados a Objetos}                  \\
        \hline
        Os dados são passivos, ou seja, certas operações limitadas podem ser automaticamente acionadas quando os dados são usados.
        Os dados são ativos, ou seja, as solicitações fazem com que os objetos executem seus métodos. & Os processos que usam dados mudam constantemente. \\
        \hline
    \end{tabular}
    \label{qua_comparabd}
    \fonte{\citeonline{Carvalho2001}}
\end{quadro}


Muitos confundem, mas existem diferenças entre \index{tabelas} e \index{quadros}.
Um quadro é formado por linhas horizontais e verticais, sendo, portanto ``fechado''. Você deverá utilizar um quadro quando o conteúdo é majoritariamente não-numérico. O número do quadro e o título vem acima do quadro, e a fonte, deve vir abaixo.
Uma tabela é formada apenas por linhas verticais, sendo, portanto ``aberta''. Você deverá utilizar uma tabela quando o conteúdo é majoritariamente numérico. O número da tabela e o título vem acima da tabela, e a fonte, deve vir abaixo, tal como no quadro.

Exemplo de tabela:

\begin{table}[!htb]
    \centering
    \caption[Resultado dos testes]{Resultado dos testes.
    \label{tab:testes}}
    \begin{tabular}{rrrrr}
        \toprule
            & Valores 1 & Valores 2 & Valores 3 & Valores 4 \\
        \midrule
            Caso 1 & 0,86 & 0,77 & 0,81 & 163 \\
            Caso 2 & 0,19 & 0,74 & 0,25 & 180 \\
            Caso 3 & 1,00 & 1,00 & 1,00 & 170 \\
        \bottomrule
    \end{tabular}
\end{table}


\section{Equações}
\label{sec:equacoes}

A transformada de Laplace é dada na \autoref{eq:laplace}, enquanto a Eq. \ref{eq:dft} apresenta a formulação da transformada discreta de Fourier bidimensional\footnote{Deve-se reparar na formatação esteticamente perfeita destas equações.}. Observe que utilizamos propositalmente duas formas distintas para referenciar as equações.

\begin{equation}
    X(s) = \int\limits_{t = -\infty}^{\infty} x(t) \, \text{e}^{-st} \, dt
    \label{eq:laplace}
\end{equation}

\begin{equation}
    F(u, v) = \sum_{m = 0}^{M - 1} \sum_{n = 0}^{N - 1} f(m, n) \exp \left[ -j 2 \pi \left( \frac{u m}{M} + \frac{v n}{N} \right) \right]
    \label{eq:dft}
\end{equation}

\section{Algoritmos}\label{sec:algoritmos}

Os \index{algoritmos} devem ser feitos segundo o modelo abaixo.

\begin{algorithm}
    \caption{Algoritmo para remoção aleatória de vértices}
    \KwIn{o número $n$ de vértices a remover, grafo original $G(V, E)$}
    \KwOut{grafo reduzido $G'(V,E)$}
    $removidos \leftarrow 0$ \\
    \While {removidos $<$ n } {
        $v \leftarrow$ Random$(1, ..., k) \in V$ \\
            \For {$u \in adjacentes(v)$} {
                remove aresta (u, v)\\
                $removidos \leftarrow removidos + 1$\\
            }
            \If {há  componentes desconectados} {
                remove os componentes desconectados\\
            }
        }
\end{algorithm}

% -----------------------------------------------------------------------------
% Novo apêndice
% -----------------------------------------------------------------------------

\chapter{Sobre as listas}
\label{chap:sobre_as_listas}

O exemplo a seguir ilustra duas listas não numeradas aninhadas, utilizando o ambiente \verb|\itemize|. Observe a indentação, bem como a mudança automática do tipo de ``\textit{bullet}'' nas listas aninhadas.

\begin{itemize}
    \item item não numerado 1
    \item item não numerado 2
    \begin{itemize}
        \item subitem não numerado 1
        \item subitem não numerado 2
        \item subitem não numerado 3
    \end{itemize}
    \item item não numerado 3
\end{itemize}

Por outro lado, o exemplo a seguir ilustra duas listas numeradas aninhadas, utilizando o ambiente \verb|\enumerate|. Observe a numeração progressiva e indentação das listas aninhadas.

\begin{enumerate}
    \item item numerado 1
    \item item numerado 2
    \begin{enumerate}
        \item subitem numerado 1
        \item subitem numerado 2
        \item subitem numerado 3
    \end{enumerate}
    \item item numerado 3
\end{enumerate}

% -----------------------------------------------------------------------------
% Novo apêndice
% -----------------------------------------------------------------------------

\chapter{Sobre as citações e chamadas de referências}
\label{chap:sobre_as_citacoes}

Citações são trechos transcritos ou informações retiradas das publicações consultadas para a realização do trabalho.
As citações são utilizadas no texto com o propósito de esclarecer, completar, embasar ou corroborar as ideias do autor.

Todas as publicações consultadas e efetivamente utilizadas (por meio de citações) devem ser listadas, obrigatoriamente, nas referências bibliográficas, de forma a preservar os direitos autorais e intelectuais.

A norma ABNT NBR:10520-2002 classifica as citações em: citações livres e citações literais.

\section{Citações livres}
\label{sec:citacoes_livres}

Nas citações livres, reproduzem-se as ideias e informações de um autor, sem, entretanto, ``copiar letra por letra'' o texto do autor. Sendo assim, não há muito a dizer sobre como fazer citações livres, exceto que há que se tomar o devido cuidado com o ``recortar e colar e modificar'' para que não se caracterize plágio.

Quanto à chamada da referência, ela pode ser feita de duas maneiras distintas, conforme o nome do(s) autor(es) façam parte do seu texto ou não. Os exemplos a seguir ilustram estas duas possibilidades.

Enquanto \citeonline{Maturana2003} defendem uma epistemologia baseada na biologia. Para os autores, é necessário rever \ldots.
Por outro lado, \citeonline{Barbosa2004} contra-argumenta afirmando que \ldots.

Acima, as chamadas de referências foram feitas com o comando \verb|\citeonline{}|, que produzirá a formatação correta, conforme a norma ABNT.

Observe que em ambos os casos anteriores, a frase fica incompleta e incompreensível caso as palavras ``Maturana e Varela'' e ``Barbosa et al.'' não sejam ``pronunciadas''. Ou seja, os nomes dos autores fazem parte da frase. Neste caso, a formatação automática da chamada de referência coloca os nomes dos autores seguido, entre parêntesis pelo ano de publicação da obra referenciada. Isso apenas no caso em que se usa o esquema autor-ano, que é \textit{padrão} neste modelo \LaTeX{}.

A segunda maneira de fazer uma chamada de referência deve ser utilizada quando se quer evitar uma interrupção na sequência do texto, o que poderia, eventualmente, prejudicar a leitura.

Assim, a citação livre é feita e imediatamente após a obra referenciada deve ser colocada entre parênteses. Porém, neste caso específico, o nome do autor deve vir em caixa alta, seguido do ano da publicação, como nos exemplos a seguir.

Há defensores da epistemologia baseada na biologia que argumentam em favor da necessidade de \ldots \cite{Maturana2003}.
Por outro lado, há os que contra-argumentam afirmando que \ldots  \cite{Barbosa2004}.

Nos dois casos imediatamente acima a chamada de referência deve ser feita com o comando \verb|\cite{chave}|, que produzirá a formatação correta, conforme a norma ABNT.

Observe que o estilo de redação das frases teve que ser modificado para torná-las compreensíveis sem a menção explícita dos nomes dos autores. Estes agora não são parte integrante da frase, ficam entre parêntesis. Neste caso, a formatação automática da chamada de referência coloca, entre parêntesis, os nomes dos autores seguido pelo ano de publicação da obra referenciada. Novamente, apenas no caso em que se usa o esquema autor-ano, que é \textit{padrão} neste modelo \LaTeX{}.

Por fim, cabe chamar a atenção para o detalhe do termo \textit{et al.} que deve ser utilizado quando o trabalho citado possui mais de três autores. Esse recurso é automatizado pelo modelo proposto.

\section{Citações literais}
\label{sec:citacoes_literais}

Nas citações literais, reproduzem-se as ideias e informações de um autor, exatamente como este a expressou, ou seja, faz-se uma ``cópia letra por letra'' do texto do autor. Sendo assim, obviamente, a obra citada deve ser referenciada, sob pena de se caracterizar plágio.

Quanto à chamada da referência, ela pode ser feita de qualquer das duas maneiras mencionadas na \autoref{sec:citacoes_livres}, conforme o nome do(s) autor(es) façam parte do seu texto ou não.

Há duas maneiras distintas de se fazer uma citação literal, conforme o trecho citado seja longo ou curto.

Quando o trecho citado é longo (4 ou mais linhas) deve-se usar um parágrafo específico para a citação, na forma de um texto recuado (4 cm da margem esquerda), com tamanho de letra menor do aquela utilizada no texto e espaçamento entrelinhas simples. Veja o exemplo abaixo.

\begin{citacao}
    Desse modo, opera-se uma ruptura decisiva entre a reflexividade filosófica, isto é a possibilidade do sujeito de pensar e de refletir, e a objetividade científica.     Encontramo-nos num ponto em que o conhecimento científico está sem consciência. Sem consciência moral, sem consciência reflexiva e também subjetiva. Cada vez mais o desenvolvimento extraordinário do conhecimento científico vai tornar menos praticável a própria possibilidade de reflexão do sujeito sobre a sua pesquisa \cite[p.~28]{Silva2000}.
\end{citacao}

Para se criar o efeito demonstrado na citação anterior, deve-se utilizar o comando:

\begin{verbatim}
    \begin{citacao}
        <citacao>
    \end{citacao}
\end{verbatim}

Acima, para a chamada da referência o comando \verb|\cite[p.~28]{Silva2000}| foi utilizado, visto que os nomes dos autores não são parte do trecho citado.

Observe ainda que foi indicado o número da página da obra citada que contém o trecho citado. A localização precisa do trecho citado deve ser indicada sempre, exceto para artigos científicos (tipicamente com poucas páginas, o que geralmente não é o caso de artigos de revisão de literatura) e outros documentos com ``poucas'' páginas.

Alternativamente, é possível construir uma frase que contenha os autores, e irá encaminhar (por assim dizer) a citação literal. Assim sendo, note que pode após a citação literal não mais aparece o nome dos autores, visto que já se encontra no texto. Veja o exemplo seguinte.

No entanto, \citeonline[p.~33]{Silva2000}, ao fazerem as suas críticas à ciência moderna, afirmam:

\begin{citacao}
    Mas o curioso é que o conhecimento científico que descobriu os meios realmente extraordinários para, por exemplo, ver aquilo que se passa no nosso sol, para tentar conceber a estrutura das estrelas extremamente distantes, e até mesmo para tentar pesar o universo, o que é algo de extrema utilidade, o conhecimento científico que multiplicou seus meios de observação e de concepção do universo, dos objetos, está completamente cego, se quiser considerar-se apenas a si próprio!
\end{citacao}

Já quando o trecho citado é curto (3 ou menos linhas) ele deve inserido diretamente no texto entre aspas. Veja os dois exemplos seguintes, cada qual utilizando uma forma de chamada de referência.

A epistemologia baseada na biologia parte do princípio de que ``assumo que não posso fazer referência a entidades independentes de mim para construir meu explicar'' \cite[p.~35]{Maturana2003}.

A epistemologia baseada na biologia de \citeonline[p.~35]{Maturana2003} parte do princípio de que ``assumo que não posso fazer referência a entidades independentes de mim para construir meu explicar''.

Finalmente, e isto vale para citações curtas ou longas, caso seja necessário inserir ou suprimir (modificar de modo geral) qualquer palavra ou frase no trecho citado literalmente, qualquer que seja a finalidade, isto deve ser feito colocando sua intervenção entre colchetes retos e deve ser indicado explicitamente ao final da citação. Veja o exemplo seguinte.

A epistemologia baseada na biologia parte do princípio de que ``assumo que não posso fazer referencia [\textit{sic}] a \underline{entidades independentes} de mim [realidade objetiva] para construir meu explicar'' \cite[p.~35, comentários e grifo nosso]{Maturana2003}.

\section{Mais detalhes sobre as chamadas de referências}
\label{sec:chamadas_referencias}

A seguir há mais exemplos dos comandos para as chamadas de referências e o resultado produzido:

\citeonline{Barbosa2004} $\Longrightarrow$ \verb|\citeonline{Barbosa2004}|

\citeonline[p.~33]{Silva2000} $\Longrightarrow$ \verb|\citeonline[p.~33]{Silva2000}|

\cite[p.~28]{Silva2000} $\Longrightarrow$ \verb|\cite[p.~28]{Silva2000}|

\cite[p.~35]{Maturana2003} $\Longrightarrow$ \verb|\cite[p.~35]{Maturana2003}|

\vspace{3em}

Há que se tomar bastante cuidado com referências cujos autores têm nomes compostos, tipo João de Souza Júnior ou Antônio José da Silva Filho. Para que a formatação seja correta, os nomes dos autores no arquivo {\ttfamily referencias.bib} deverá ser cadastrado de uma forma específica.

Os exemplos abaixo ilustram a formatação correta:

\cite[p.~28]{vanGELDER1998} $\Longrightarrow$ \verb|\cite[p.~28]{vanGELDER1998}|

\citeonline[p.~28]{vanGELDER1998} $\Longrightarrow$ \verb|\citeonline[p.~28]{vanGELDER1998}|

\vspace{3em}

Observe ainda o caso em que é feita duas citações juntas \cite{Silva2000, Barbosa2004} e como citar endereços de páginas da Internet \cite{IRL2014}.

% -----------------------------------------------------------------------------
% Novo apêndice
% -----------------------------------------------------------------------------

\chapter{Sobre as referências bibliográficas}
\label{chap:sobre_as_referencias_bibliograficas}

A bibliografia é feita no padrão \textsc{Bib}\TeX{}. As referências são colocadas em um arquivo separado chamado {\ttfamily referencias.bib}. Os elementos de cada item bibliográfico que devem constar nas referências bibliográficas são apresentados a seguir. Tais referências bibliográficas devem seguir a norma \citeonline{NBR6023:2002} da ABNT\footnote{As normas técnicas da ABNT não são gratuitas.}.

\section{Entradas de referências}
\label{sec:entradas_de_referencias}

Entradas são objetos de citação bibliográficas. Dito de outra forma, são as categorias dos tipos de documentos e materiais componentes da bibliografia. A classe abn\TeX{} define as seguintes entradas:

\begin{verbatim}
    @book
    @inbook
    @article
    @phdthesis
    @mastersthesis
    @monography
    @techreport
    @manual
    @proceedings
    @inproceedings
    @journalpart
    @booklet
    @patent
    @unpublished
    @misc
\end{verbatim}

Cada entrada é formatada pelo modelo de uma forma específica. Algumas entradas foram introduzidas especificamente para atender à norma \citeonline{NBR6023:2002}, são elas: \verb|@monography|, \verb|@journalpart|,\verb|@patent|. As demais entradas são padrão \textsc{Bib}\TeX{}. Para maiores detalhes, refira-se a \citeonline{abnTeX22014d}, \citeonline{abnTeX22014b}, \citeonline{abnTeX22014c}.

A entrada \verb|@monography| é utilizada para cadastrar referências a trabalhos de conclusão de curso, monografias de cursos de especialização (pós-graduação \textit{lato sensu}), e outros trabalhos monográficos, exceto dissertação de mestrado e tese de doutorado. Eu particularmente, não considero que a formatação deste tipo de entrada está adequado. Para um trabalho de conclusão de curso (TCC) de curso de graduação, que deveria ser formatado como ``[\ldots] Trabalho de Conclusão de Curso (Bacharelado em Engenharia de Computação) [\ldots]''; no entanto o uso de \verb|@monography| irá produzir ``[\ldots] Monografia (Bacharelado em Engenharia de Computação) [\ldots]''. A própria  \citeonline{NBR6023:2002}, na seção 8.11.4, apresenta um exemplo com a formatação diferente daquela proporcionada por \citeonline{abnTeX22014d}.

A entrada \verb|@journalpart| é utilizada, conforme diz o manual \cite{abnTeX22014d}, para cadastrar referências e formatar partes de periódicos. Não fica claro o que se quer dizer com partes de journal. Em alguns casos, tais partes são artigos - e portanto, deveriam ser registradas como \verb|@article| - noutros casos, parece serem matérias ou textos em revistas ou jornais (não científicos). Salvo melhor juízo, me parece que esta entrada deve ser utilizada apenas neste último contexto.

A entrada \verb|@patent| é utilizada, obviamente, para cadastrar referências a patentes.

Todavia, o fato é que a normalização de referências conforme a norma \citeonline{NBR6023:2002} requer que muitos dos campos do \textsc{Bib}\TeX{} sejam adaptados. Sendo mais explícito, ao baixar um arquivo {\ttfamily .bib} de um trabalho, principalmente ser for internacional, e inserí-lo ``\textit{as is}'' em suas referências, há grande chance dessa referência ser formatada de modo errado, no que concerne à norma \citeonline{NBR6023:2002}. Isso é especialmente válido em alguns tipos de documentos de largo uso no meio acadêmico afim às áreas de Ciências Exatas, da Terra e Engenharias.

Diante disso, para evitar erros de formatação, o correto é após baixar o arquivo {\ttfamily .bib} de um trabalho, editá-lo com um editor de textos (usando a codificação UTF8), para verificar se os campos descritores que o \textit{publishers} original utilizou são aqueles requeridos pela norma ABNT.

Neste contexto, e para esta finalidade, nas seções seguintes é apresentado uma série de exemplos, quase todos, utilizados como exemplos na própria norma \citeonline{NBR6023:2002}. Para detalhes dos campos utilizados confira o arquivo {\ttfamily referencias.bib}. Deve-se estar atento para o fato de que o uso de um sistema de gerenciamento de referências para abrir e/ou editar o arquivo {\ttfamily referencias.bib}, pode ocultar campos utilizados pela norma ABNT e, por outro lado, exibir campos não utilizados por ela. Ou seja, o aplicativo deve ser configurado adequadamente para exibir \textbf{todos os campos}, mesmo os opcionais.

\section{Notas de rodapé}
\label{sec:notas_de_rodape}

A norma \citeonline{NBR10520:2002} classifica as notas de rodapé em duas categorias: notas explicativas\footnote{É o tipo mais comum de notas que destacam, explicam e/ou complementam o que foi dito no corpo do texto, como esta nota de rodapé, por exemplo.} e notas de referências. Já as notas de referências, como o próprio nome já indica, são utilizadas para colocar referências e/ou chamadas de referências sob certas condições.

\subsection{Notas de referências: uso de idem, ibidem, opus citatus e outros}
\label{subsec:notas_de_referencias}

Como indica o próprio nome, as notas de referências se prestam como recurso auxiliar para referenciação de bibliografia, e seu uso e aplicação são descritos na norma \citeonline{NBR10520:2002}.

Estes recursos se referem ao uso de certas expressões consagradas para facilitar a elaboração de referencias. São eles:

\begin{itemize}
    \item idem = mesmo autor,
    \item ibidem = mesma obra,
    \item opus citatum = obra citada,
    \item locus citatum = no lugar citado,
    \item passim = aqui e alí,
    \item cf = confira,
    \item et sequentia = e sequência.
\end{itemize}

Observe que estes recursos não se adequam para serem utilizados em listas de referências bibliográficas, nem tampouco no corpo do texto. Assim, devem ser utilizados apenas nas notas de referência posicionadas no rodapé \cite[p.~6]{abnTeX22014c}, quando se referem a uma referência já feita anteriormente no corpo do texto. Ademais, essas expressões fazem sentido apenas quando aplicadas a citações de uma única referência por vez. Enfim, trata-se mais de um recurso estilístico do que algo de primeira necessidade, pelo menos para o tipo de documento usualmente elaborados nas área de Ciências Exatas, da Terra e Engenharias.

Veja o uso desses tipos de expressões nos exemplos seguintes:

\Idem[p.~21]{abnTeX22014d}

\Ibidem[p.~7]{abnTeX22014c}

\opcit[p.~9]{abnTeX22014c}

\passim{abnTeX22014c}

\cfcite[p.~3]{abnTeX22014b}

\etseq[p.~6]{abnTeX22014c}

\section{Datas em referências}
\label{sec:datas_em_referencias}

Quando as chamadas de referências são feitas no modelo autor-ano, como é o caso deste modelo, é evidente que o autor e sobretudo o ano adquirem papel de destaque. No caso da data de publicação, esta deve sempre estar presente (é elemento essencial) e indicada em algarismos arábicos.

A norma ABNT não permite o uso de expressões do tipo ``sem data'' (``[s.d]``) para indicar que não se sabe a data de publicação de certa referência bibliográfica. Assim sendo, quando a data não puder ser indicada precisamente, deve-se registrar uma data aproximada entre colchetes, conforme descrito a seguir:

[1971 ou 1972] \ \ \ \ \ um ano ou outro,

[1969?] \ \ \ \ \ ano provável,

[1973] \ \ \ \ \ ano certo, não indicada no item,

[entre 1906 e 1912] \ \ \ \ \ use intervalos menores de 20 anos,

[ca. 1960] \ \ \ \ \ \textit{circa} de ... ( data aproximada),

[197-] \ \ \ \ \ década certa,

[197-?] \ \ \ \ \ década provável,

[18--] \ \ \ \ \ século certo,

[18--?] \ \ \ \ \ século provável.

\end{apendicesenv}
            % Apêndices
% ------------------------------------------------------------------------------
% Anexos
% ------------------------------------------------------------------------------

\begin{anexosenv}
    \partanexos

    % --------------------------------------------------------------------------
    % Primeiro anexo
    % --------------------------------------------------------------------------

    \chapter{Nome do anexo}
    \label{chap_anexo_a}

    Lembre-se que a diferença entre apêndice e anexo diz respeito à autoria do texto e/ou material ali colocado.

    Caso o material ou texto suplementar ou complementar seja de sua autoria, então ele deverá ser colocado como um apêndice.
    Porém, caso a autoria seja de terceiros, então o material ou texto deverá ser colocado como anexo.

    Caso seja conveniente, podem ser criados outros anexos para o seu trabalho acadêmico.
    Basta recortar e colar este trecho neste mesmo documento.

    Organize seus anexos de modo a que, em cada um deles, haja um único tipo de conteúdo.
    Isso facilita a leitura e compreensão para o leitor do trabalho. É para ele que você escreve.

\end{anexosenv}
               % Anexos
\imprimirindiceremissivo                            % Índice Remissivo

\end{document}
