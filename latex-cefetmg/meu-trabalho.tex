% -----------------------------------------------------------------------------
%       Centro Federal de Educação Tecnológica de Minas Gerais - CEFET-MG
%
%   Modelo de trabalho acadêmico monográfico de acordo com as normas da ABNT
%   (Tese de Doutorado, Dissertação de Mestrado ou Projeto de Qualificação)
%
%    Autores: Cristiano Fraga G. Nunes <cfgnunes@gmail.com>
%             Henrique E. Borges <henrique@lsi.cefetmg.br>
%             Denise de Souza <densouza@gmail.com>
%             Lauro César <https://code.google.com/p/abntex2/>
%
% -----------------------------------------------------------------------------

\documentclass[%
    %twoside,                   % Impressão em frente (anverso) e verso. Oposto a oneside
    oneside,                    % Impressão apenas no anverso. Oposto a twoside
]{cefetmg}

\usepackage[%
    alf,
    abnt-emphasize=bf,
    bibjustif,
    recuo=0cm,
    abnt-doi=expand,            % Expande um endereço iniciado com doi: para http://dx.doi.org/
    abnt-url-package=url,       % Utiliza o pacote url
    abnt-refinfo=yes,           % Utiliza o estilo bibliográfico abnt-refinfo
    abnt-etal-cite=3,
    abnt-etal-list=3,
    abnt-thesis-year=final
]{abntex2cite}                  % Configura as citações bibliográficas conforme a norma ABNT

% -----------------------------------------------------------------------------
% Pacotes utilizados
% -----------------------------------------------------------------------------
\usepackage[utf8]{inputenc}                                 % Codificação do documento
\usepackage[T1]{fontenc}                                    % Seleção de código de fonte
\usepackage{booktabs}                                       % Réguas horizontais em tabelas
\usepackage{color, colortbl}                                % Controle das cores
\usepackage{float}                                          % Necessário para tabelas/figuras em ambiente multi-colunas
\usepackage{graphicx}                                       % Inclusão de gráficos e figuras
\usepackage{icomma}                                         % Uso de vírgulas em expressões matemáticas
\usepackage{indentfirst}                                    % Indenta o primeiro parágrafo de cada seção
\usepackage{microtype}                                      % Melhora a justificação do documento
\usepackage{multirow, array}                                % Permite tabelas com múltiplas linhas e colunas
\usepackage{subeqnarray}                                    % Permite subnumeração de equações
\usepackage{verbatim}                                       % Permite apresentar texto tal como escrito no documento, ainda que sejam comandos Latex
\usepackage{amsfonts, amssymb, amsmath}                     % Fontes e símbolos matemáticos
\usepackage[algoruled, portuguese]{algorithm2e}             % Permite escrever algoritmos em português
\usepackage[scaled]{helvet}                                 % Usa a fonte Helvetica
%\usepackage{times}                                         % Usa a fonte Times
%\usepackage{palatino}                                      % Usa a fonte Palatino
%\usepackage{lmodern}                                       % Usa a fonte Latin Modern
%\usepackage[bottom]{footmisc}                              % Mantém as notas de rodapé sempre na mesma posição
%\usepackage{ae, aecompl}                                   % Fontes de alta qualidade
%\usepackage{latexsym}                                      % Símbolos matemáticos
%\usepackage{lscape}                                        % Permite páginas em modo "paisagem"
%\usepackage{picinpar}                                      % Dispor imagens em parágrafos
%\usepackage{scalefnt}                                      % Permite redimensionar tamanho da fonte
%\usepackage{subfig}                                        % Posicionamento de figuras
%\usepackage{upgreek}                                       % Fonte letras gregas

% Redefine a fonte para uma fonte similar a Arial (fonte Helvetica)
\renewcommand*\familydefault{\sfdefault}

% -----------------------------------------------------------------------------
% Configurações de aparência do PDF final
% -----------------------------------------------------------------------------
\makeatletter
\hypersetup{%
    portuguese,
    colorlinks=true,            % true: "links" coloridos; false: "links" em caixas de texto
    linkcolor=blue,             % Define cor dos "links" internos
    citecolor=blue,             % Define cor dos "links" para as referências bibliográficas
    filecolor=blue,             % Define cor dos "links" para arquivos
    urlcolor=blue,              % Define a cor dos "hiperlinks"
    breaklinks=true,
    pdftitle={\@title},
    pdfauthor={\@author},
    pdfkeywords={abnt, latex, abntex, abntex2}
}
\makeatother

% Altera o aspecto da cor azul
\definecolor{blue}{RGB}{41,5,195}

% Redefinição de labels
\renewcommand{\algorithmautorefname}{Algoritmo}
\def\equationautorefname~#1\null{Equa\c c\~ao~(#1)\null}

% Cria o índice remissivo
\makeindex

% Hifenização de palavras que não estão no dicionário
\hyphenation{%
    qua-dros-cha-ve
    Kat-sa-gge-los
}

% -----------------------------------------------------------------------------
% Inclui os arquivos do trabalho acadêmico
% -----------------------------------------------------------------------------

% Insere e constrói alguns elementos pré-textuais para gerar capa, folha de rosto e folha de aprovação
% -----------------------------------------------------------------------------
% Capa
% -----------------------------------------------------------------------------


% -----------------------------------------------------------------------------
% ATENÇÃO:
% Caso algum campo não se aplique ao seu documento - por exemplo, em seu trabalho
% não houve coorientador - não comente o campo, apenas deixe vazio, assim: \campo{}
% -----------------------------------------------------------------------------


% -----------------------------------------------------------------------------
% Dados do trabalho acadêmico
% -----------------------------------------------------------------------------

\titulo{Título do Trabalho}
%\title{Title in English}
\subtitulo{Subtítulo do trabalho}
\autor{Nome completo do autor}
\local{Belo Horizonte}
\data{Junho de 2016}            % Normalmente se usa apenas mês e ano


% -----------------------------------------------------------------------------
% Natureza do trabalho acadêmico
% Use apenas uma das opções: Tese (p/ Doutorado), Dissertação (p/ Mestrado) ou
% Projeto de Qualificação (p/ Mestrado ou Doutorado), Trabalho de Conclusão de
% Curso (Graduação)
% -----------------------------------------------------------------------------

\projeto{Projeto de Qualificação}


% -----------------------------------------------------------------------------
% Título acadêmico
% Use apenas uma das opções:
%   Se a natureza for Tese, coloque Doutor
%   Se a natureza for Dissertação, coloque Mestre
%   Se a natureza for Projeto de Qualificação, coloque Mestre ou Doutor conforme o caso
% Se a natureza for Trabalho de Conclusão de Curso, coloque Bacharel
% -----------------------------------------------------------------------------

\tituloAcademico{Doutor}


% -----------------------------------------------------------------------------
% Área de concentração e linha de pesquisa
%   OBS: indique o nome da área de concentração e da linha de pesquisa do Programa de Pós-graduação
% nas quais este trabalho se insere
% Se a natureza for Trabalho de Conclusão de Curso, deixe ambos os campos vazios
% -----------------------------------------------------------------------------

\areaconcentracao{Modelagem Matemática e Computacional}
\linhapesquisa{Sistemas Inteligentes}


% -----------------------------------------------------------------------------
% Dados da instituição
% OBS: a logomarca da instituição deve ser colocada na mesma pasta que foi colocada o documento
% principal com o nome de "logoInstituicao". O formato pode ser: pdf, jpf, eps
% Se a natureza for Trabalho de Conclusão de Curso, coloque em "programa' o nome do curso de graduação
% -----------------------------------------------------------------------------

\instituicao{Centro Federal de Educação Tecnológica de Minas Gerais}
\programa{Programa de Pós-graduação em Modelagem Matemática e Computacional}
%\programa{Curso de Engenharia de Computação}
\logoinstituicao{0.2}{./04-figuras/logo-instituicao}                  % \logoinstituicao{<escala>}{<nome do arquivo>}


% -----------------------------------------------------------------------------
% Dados do(s) orientador(es)
% -----------------------------------------------------------------------------

\orientador{Nome do orientador}
%\orientador[Orientadora:]{Nome da orientadora}
\instOrientador{Instituição do orientador}

\coorientador{Nome do coorientador}
%\coorientador[Coorientadora:]{Nome da coorientadora}
\instCoorientador{Instituição do coorientador}

% -----------------------------------------------------------------------------
% Folha de Rosto
% -----------------------------------------------------------------------------

% Trabalho de Conclusão de Curso
%\preambulo{{\imprimirprojeto} apresentado ao Curso de Engenharia de Computação do Centro Federal de Educação Tecnológica de Minas Gerais, como requisito parcial para a obtenção do título de {\imprimirtituloAcademico} em Engenharia de Computação.}

% Projeto de qualificação de Mestrado ou Doutorado
\preambulo{{\imprimirprojeto} apresentado ao Programa de \mbox{Pós-graduação} em Modelagem Matemática e Computacional do Centro Federal de Educação Tecnológica de Minas Gerais, como requisito parcial para a obtenção do título de {\imprimirtituloAcademico} em Modelagem Matemática e Computacional.}

% Dissertação de Mestrado
%\preambulo{{\imprimirprojeto} apresentada ao Programa de \mbox{Pós-graduação} em Modelagem Matemática e Computacional do Centro Federal de Educação Tecnológica de Minas Gerais, como requisito parcial para a obtenção do título de {\imprimirtituloAcademico} em Modelagem Matemática e Computacional.}

% Tese de Doutorado
%\preambulo{{\imprimirprojeto} apresentada ao Programa de \mbox{Pós-graduação} em Modelagem Matemática e Computacional do Centro Federal de Educação Tecnológica de Minas Gerais, como requisito parcial para a obtenção do título de {\imprimirtituloAcademico} em Modelagem Matemática e Computacional.}

% -----------------------------------------------------------------------------
% Edite este arquivo comentando as linhas que não se aplicam ao tipo de
% documento acadêmico pretendido.
% -----------------------------------------------------------------------------

% -----------------------------------------------------------------------------
% Folha de Aprovação
% -----------------------------------------------------------------------------

\textopadraofolhadeaprovacao{Esta folha deverá ser substituída pela cópia digitalizada da folha de aprovação fornecida.}

% -----------------------------------------------------------------------------
% Este documento foi mantido apenas para preservar a paginação do trabalho
% acadêmico final, após a inserção da folha de aprovação fornecida
% -----------------------------------------------------------------------------


\begin{document}

% Insere os elementos pré-textuais
\pretextual
\imprimircapa                                               % Comando para imprimir Capa
\imprimirfolhaderosto{}                                     % Comando para imprimir Folha de rosto
\imprimirfolhadeaprovacao{}                                 % Comando para imprimir Folha de aprovação
%
% Documento: Dedicatória
%

\begin{dedicatoria}

Espaço reservado para dedicatória.
Inserir seu texto aqui...

\end{dedicatoria}
           % Dedicatória
% -----------------------------------------------------------------------------
% Agradecimentos
% -----------------------------------------------------------------------------

\begin{agradecimentos}

    Edite e coloque aqui os agradecimentos às pessoas e/ou instituições que contribuíram para a realização do trabalho.

    É obrigatório o agradecimento às instituições de fomento à pesquisa que financiaram total ou parcialmente o trabalho, inclusive no que diz respeito à concessão de bolsas.

\end{agradecimentos}
        % Agradecimentos
%
% Documento: Epígrafe
%

\begin{epigrafe}
    \vspace*{\fill}
	\begin{flushright}
		\textit{``O fator decisivo para vencer o maior obstáculo é, invariavelmente, ultrapassar o obstáculo anterior.'' (Henry Ford)}
	\end{flushright}
\end{epigrafe}
              % Epígrafe
\include{./01-elementos-pre-textuais/resumo-pt}             % Resumo na língua vernácula
% -----------------------------------------------------------------------------
% Abstract
% -----------------------------------------------------------------------------

\begin{resumo}[Abstract]

    Translation of the abstract into english, possibly adapting or slightly changing the text in order to adjust it to the grammar of Standard English.
    Try to stay within the limit of: 500 word for a PhD Thesis;
    250 words for a Master Dissertation;
    200 words for a Qualifying Research Project.

    \par\vspace{\baselineskip}

    \textbf{Keywords}: Latex model. Academic work. ABNT standards. Another word.
\end{resumo}

% Em uma Tese de Doutorado o resumo deve conter, no máximo, 500 palavras.
% Em uma Dissertação de Mestrado o resumo deve conter, no máximo, 250 palavras.
% Em um Projeto de Qualificação o resumo deve conter, no máximo, 200 palavras.

% -----------------------------------------------------------------------------
% O restante da formatação deve manter-se igual ao do resumo em português,
% por exemplo, um único parágrafo.
% -----------------------------------------------------------------------------
             % Resumo em língua inglesa
% -----------------------------------------------------------------------------
% Lista de Figuras
% -----------------------------------------------------------------------------

\pdfbookmark[0]{\listfigurename}{lof}
\listoffigures*
\cleardoublepage

% -----------------------------------------------------------------------------
% Não é necessário editar este arquivo. A lista é gerada automaticamente.
% -----------------------------------------------------------------------------
         % Lista de figuras
% -----------------------------------------------------------------------------
% Lista de Tabelas
% -----------------------------------------------------------------------------

\pdfbookmark[0]{\listtablename}{lot}
\listoftables*
\cleardoublepage

% -----------------------------------------------------------------------------
% Não é necessário editar este arquivo. A lista é gerada automaticamente.
% -----------------------------------------------------------------------------
         % Lista de tabelas
%
% Documento: Lista de quadros
%

\pdfbookmark[0]{\listofquadrosname}{loq}
\listofquadros*
\cleardoublepage
         % Lista de quadros
% -----------------------------------------------------------------------------
% Lista de Algoritmos
% -----------------------------------------------------------------------------

\newcommand{\algoritmoname}{Algoritmo}
\renewcommand{\listalgorithmcfname}{Lista de Algoritmos}

\floatname{algocf}{\algoritmoname}
\newlistof{listofalgoritmos}{loa}{\listalgoritmoname}
\newlistentry{algocf}{loa}{0}

\counterwithout{algocf}{chapter}
\renewcommand{\cftalgocfname}{\algoritmoname\space}
\renewcommand*{\cftalgocfaftersnum}{\hfill--\hfill}

\pdfbookmark[0]{\listalgorithmcfname}{loa}
\listofalgorithms
\cleardoublepage

% -----------------------------------------------------------------------------
% Este arquivo não necessita de ser editado. A lista é gerada automaticamente.
% -----------------------------------------------------------------------------
      % Lista de algoritmos
% -----------------------------------------------------------------------------
% Lista de Siglas
% -----------------------------------------------------------------------------

\begin{siglas}
    \item[ABNT] Associação Brasileira de Normas Técnicas
    \item[DECOM] Departamento de Computação
\end{siglas}

% -----------------------------------------------------------------------------
% Edite a lista acima para definir "todos" os acrônimos e siglas utilizados neste trabalho
% -----------------------------------------------------------------------------
          % Lista de abreviaturas e siglas
% -----------------------------------------------------------------------------
% Lista de Símbolos
% -----------------------------------------------------------------------------

\begin{simbolos}
    \item[$ \Gamma $] Letra grega Gama
    \item[$ \lambda $] Comprimento de onda
    \item[$ \in $] Pertence
\end{simbolos}

% -----------------------------------------------------------------------------
% Edite a lista acima para definir "todos" os símbolos utilizados neste trabalho
% -----------------------------------------------------------------------------
        % Lista de símbolos
% -----------------------------------------------------------------------------
% Sumário
% -----------------------------------------------------------------------------

\pdfbookmark[0]{\contentsname}{toc}
\tableofcontents*
\cleardoublepage

% -----------------------------------------------------------------------------
% Este arquivo não necessita de ser editado. O sumário é gerado automaticamente.
% -----------------------------------------------------------------------------
               % Sumário

% Insere os elementos textuais
\textual
%
% Documento: Introdução
%

\chapter{Introdução}\label{chap:introducao}

O presente documento é um exemplo de uso do estilo de formatação LATEX elaborado para atender às Normas para Elaboração de Trabalhos Acadêmicos. O estilo de formatação {\ttfamily abntex2-cefetmg.cls} tem por base o pacote ABNTEX -- cuja leitura da documentação \cite{abnTeX2009} é fortemente sugerida.

Para melhor entendimento do uso do estilo de formatação, aconselha-se que o potencial usuário analise os comandos existentes no arquivo {\ttfamily main.tex} e os resultados obtidos no arquivo {\ttfamily main.pdf} depois do processamento pelo software LATEX + BIBTEX \cite{LaTeX2009,BibTeX2009}. Recomenda-se a consulta ao material de referência do software para a sua correta utilização \cite{Lamport1986,Buerger1989,Kopka2003,Mittelbach2004}.

\section{Motivação}
\label{sec:motivacao}

Uma das principais vantagens do uso do estilo de formatação para LATEX é a formatação \textit{automática} dos elementos que compõem um documento acadêmico, tais como capa, folha de rosto, dedicatória, agradecimentos, epígrafe, resumo, abstract, listas de figuras, tabelas, siglas e símbolos, sumário, capítulos, referências, etc. Outras grandes vantagens do uso do LATEX para formatação de documentos acadêmicos dizem respeito à facilidade de gerenciamento de referências cruzadas e bibliográficas, além da formatação -- inclusive de equações matemáticas -- correta e esteticamente perfeita.

\section{Caracterização do Problema}
\label{sec:caracProblema}

Inserir seu texto aqui...

\section{Objetivos}
\label{sec:objetivos}

\subsection{Objetivo Geral}
\label{subsec:objetivoGeral}

Prover um modelo de formatação LATEX que atenda às normas da instituição atual e às normas brasileiras.

\subsection{Objetivos Específicos}
\label{subsec:objetivosEspecificos}

\begin{itemize}
	\item Obter documentos acadêmicos automaticamente formatados com correção e perfeição estética.
	\item Desonerar autores da tediosa tarefa de formatar documentos acadêmicos, permitindo sua concentração no conteúdo do mesmo.
	\item Desonerar orientadores e examinadores da tediosa tarefa de conferir a formatação de documentos acadêmicos, permitindo sua concentração no conteúdo do mesmo.
\end{itemize}


\section{Organização do Documento}
\label{sec:organizacaoDocumento}

Inserir seu texto aqui...

\section{Justificativa}
\label{sec:justificativa}

Inserir seu texto aqui...
                % Introdução
% -----------------------------------------------------------------------------
% Trabalhos Relacionados
% -----------------------------------------------------------------------------

\chapter{Trabalhos Relacionados}
\label{chap:trabRelac}

Cada capítulo deve conter uma pequena introdução (tipicamente, um ou dois parágrafos), em seção não numerada, que deve deixar claro o objetivo e o que será discutido no capítulo, bem como a organização do capítulo.
    % Trabalhos relacionados
% ------------------------------------------------------------------------------
% Fundamentação Teórica
% ------------------------------------------------------------------------------

\chapter{Fundamentação Teórica}
\label{chap_fundamentacao_teorica}

É uma boa prática iniciar cada novo capítulo com uma breve texto introdutório (tipicamente, dois ou três parágrafos) que deve deixar claro o quê será discutido no capítulo, bem como a organização do capítulo.
Também servirá ao propósito de ``amarrar'' ou ``alinhavar'' o conteúdo deste capítulo com o conteúdo do capítulo imediatamente anterior.
     % Fundamentação teórica
%
% Documento: Metodologia
%

\chapter{Metodologia}

Inserir seu texto aqui...

\section{Delineamento da pesquisa}

Inserir seu texto aqui...

\section{Coleta de dados}

Inserir seu texto aqui...

               % Metodologia
% -----------------------------------------------------------------------------
% Resultados
% -----------------------------------------------------------------------------

\chapter{Análise e Discussão dos Resultados}

Cada capítulo deve conter uma pequena introdução (tipicamente, um ou dois parágrafos), em seção não numerada, que deve deixar claro o objetivo e o que será discutido no capítulo, bem como a organização do capítulo.

\section{Título da seção}
\label{sec:titSecResult}

Inserir seu texto aqui...
                % Resultados
% -----------------------------------------------------------------------------
% Conclusão
% -----------------------------------------------------------------------------

\chapter{Conclusão}
\label{chap:conclusao}

Procure fazer uma análise crítica de seu trabalho, destacando os principais resultados e as contribuições deste trabalho para a área de pesquisa.

\section{Trabalhos Futuros}
\label{sec:trabalhosFuturos}

Também deve indicar, se possível e/ou conveniente, como este trabalho pode ser estendido ou aprimorado.

\section{Considerações Finais}
\label{sec:consideracoesFinais}

As derradeiras palavras para encerramento do seu trabalho acadêmico.

% -----------------------------------------------------------------------------
% Observação: A norma ABNT estabelece que em qualquer tipo de trabalho
% acadêmico monográfico deve haver um capítulo de conclusão
% -----------------------------------------------------------------------------
                 % Conclusão

% Insere os elementos pós-textuais
\postextual
% -----------------------------------------------------------------------------
% Referências
% -----------------------------------------------------------------------------

% -----------------------------------------------------------------------------
% Carrega o arquivo "base-referencias.bib" e extrai automaticamente as referências citadas
% -----------------------------------------------------------------------------

\bibliography{./referencias}{}
\bibliographystyle{abntex2-alf} % Define o estilo ABNT para formatar a lista de referências

% -----------------------------------------------------------------------------
% Este arquivo não necessita de ser editado.
% -----------------------------------------------------------------------------
           % Referências
% -----------------------------------------------------------------------------
% Apêndices
% -----------------------------------------------------------------------------

\begin{apendicesenv}
\partapendices

% -----------------------------------------------------------------------------
% Primeiro apêndice
% -----------------------------------------------------------------------------

\chapter{Nome do apêndice} % Edite para alterar o título deste apêndice
\label{chap:apendiceA}

Lembre-se que a diferença entre apêndice e anexo diz respeito à autoria do texto e/ou material ali colocado.

Caso o material ou texto suplementar ou complementar seja de sua autoria, então ele deverá ser colocado como um apêndice. Porém, caso a autoria seja de terceiros, então o material ou texto deverá ser colocado como anexo.

Caso seja conveniente, podem ser criados outros apêndices para o seu trabalho acadêmico. Basta recortar e colar este trecho neste mesmo documento. Lembre-se de alterar o "label"{} do apêndice.

Não queira colocar tudo que é complementar em um único apêndice. Organize seus apêndices de modo a que, em cada um deles, haja um único tipo de conteúdo. Isso facilita a leitura e compreensão para o leitor do trabalho. É para ele que você escreve.

% -----------------------------------------------------------------------------
% Novo apêndice
% -----------------------------------------------------------------------------

\chapter{Estrutura de trabalhos acadêmicos}
\label{chap:apEstrTrabAcad}

Quanto à estrutura do trabalho acadêmico, esta varia sobremaneira, a depender da conveniência do autor e seu(s) respectivo(s) orientador(es). No entanto, de acordo com as normas ABNT, alguns elementos são obrigatórios.

A título de sugestão, e apenas isso, a \autoref{fig:estrutura-projeto-qualificacao} apresenta uma estrutura para um projeto de qualificação de mestrado ou doutorado, conforme a norma \citeonline{NBR14724:2011}.

\begin{figure}[!htb]
    \centering
    \caption{Estrutura sugerida de um Projeto de Qualificação para os cursos de Mestrado ou Doutorado}
    \includegraphics[width=0.5\textwidth]{./04-figuras/estrutura-projeto-qualificacao}
    \label{fig:estrutura-projeto-qualificacao}
\end{figure}

Já a \autoref{fig:estrutura-tese-dissertacao} apresenta uma estrutura para uma tese de doutorado ou dissertação de mestrado, conforme a norma \citeonline{NBR14724:2011}.

Cabe ressaltar que, em todas as figuras, os elementos obrigatórios estão destacados em vermelho, os demais são opcionais.

\begin{figure}[!htb]
    \centering
    \caption{Estrutura sugerida de uma Tese de Doutorado ou Dissertação de Mestrado}
    \includegraphics[width=0.5\textwidth]{./04-figuras/estrutura-tese-dissertacao}
    \label{fig:estrutura-tese-dissertacao}
\end{figure}

Observe que a estrutura de um projeto de qualificação é muito similar à da tese ou dissertação. A única diferença existente é que num projeto de qualificação o autor certamente terá, via de regra, apenas resultados parciais e preliminares. Além disso, estando o trabalho ainda em andamento, há que se apresentar um cronograma de trabalho que evidencie que o mesmo poderá ser concluído dentro dos prazos estabelecidos pelo programa.

Por fim, como foi dito, este  \emph{template} pode ser utilizado para outros trabalhos acadêmicos. Neste caso, a \autoref{fig:estrutura-projeto-pesquisa} apresenta uma sugestão de projeto de pesquisa a ser submetido ao programa para fins de admissão ao mesmo, conforme a norma \citeonline{NBR15287:2005}.

\begin{figure}[!htb]
    \centering
    \caption{Estrutura sugerida de um projeto de pesquisa para admissão ao PPGMMC}
    \includegraphics[width=0.6\textwidth]{./04-figuras/estrutura-projeto-pesquisa}
    \label{fig:estrutura-projeto-pesquisa}
\end{figure}

Você deverá editar o arquivo principal {\ttfamily meuTrabalhoAcademico.tex} para fazer os ajustes necessários, reiterando que as estruturas apresentadas são mera sugestão.

A inclusão de reticências (\ldots) no texto deverá ser feita através de um comando especial denominado \verb|\ldots| \cite{LaTeX2014}. Assim esse comando deverá ser utilizado ao invés da digitação de três pontos.

Para melhor entendimento do uso do estilo de formatação, aconselha-se que o potencial usuário analise os comandos existentes no arquivo {\ttfamily main.tex} e os resultados obtidos no arquivo {\ttfamily main.pdf} depois do processamento pelo software \LaTeX{} + \textsc{Bib}\TeX{} \cite{LaTeX2014,BibTeX2014}.
Recomenda-se a consulta ao material de referência do software para a sua correta utilização \cite{Lamport1986,Buerger1989,Kopka2003,Mittelbach2004}.

Finalmente, este modelo apresenta um arquivo {\ttfamily makefile} para agilizar a compilação do documento \LaTeX{} e do \textsc{Bib}\TeX{}. portanto, para gerar o documento final no formato PDF, basta apenas executar o comando {\ttfamily make all} no linux. Para limpar os arquivos temporários, basta digitar o comando {\ttfamily make clean}.

O estilo de documento utilizado é o {\ttfamily abntex2}.
Através desse estilo a constituição do documento torna-se facilitada, uma vez que o mesmo possui comandos especiais para auxiliar a distribuição/definição das diversas partes constituintes do projeto.
Esse estilo é baseado nas normas da ABNT\index{ABNT}.

Maiores detalhes relacionados aos comandos existentes no estilo poderão ser adquiridos através da documentação disponível no site \href{https://code.google.com/p/abntex2/}{https://code.google.com/p/abntex2/} \cite{abnTeX22014b}.

Uma das principais vantagens do uso do estilo de formatação para \LaTeX{}  é a formatação \textit{automática} dos elementos que compõem um documento acadêmico, tais como capa, folha de rosto, dedicatória, agradecimentos, epígrafe, resumo, abstract, listas de figuras, tabelas, siglas e símbolos, sumário, capítulos, referências, etc.

% -----------------------------------------------------------------------------
% Novo apêndice
% -----------------------------------------------------------------------------

\chapter{Sobre as ilustrações}
\label{chap:apSobreIlust}

A seguir ilustra-se a forma de incluir ilustrações no corpo do texto. Pela norma figuras, tabelas, quadros, equações, quadros, algoritmos, diagrama, etc. são tipos específicos de ilustrações. As ilustrações (pelo menos alguns tipos específicos) serão indexadas automática em suas respectivas listas.

A numeração sequencial de figuras, tabelas e equações ocorre de modo automático.

Referências cruzadas são obtidas através dos comandos \verb|\label{}| e \verb|\ref{}|. Por exemplo, não é necessário saber que o número de certo capítulo é \ref{chap:fundamentacaoTeorica} para colocar o seu número no texto. Alternativamente se pode usar desta forma: \autoref{chap:fundamentacaoTeorica}. Isto facilita muito a inserção, remoção ou relocação de elementos numerados no texto (fato corriqueiro na escrita e correção de um documento acadêmico) sem a necessidade de renumerá-los todos.

\section{Figuras}
\label{sec:figuras}

Abaixo é apresentado um exemplo de figura.

A \autoref{fig:kdtree} aparece automaticamente na lista de figuras.

Para uso avançado de imagens no \LaTeX{}, recomenda-se a consulta de literatura especializada \cite{Goossens2007}.

\begin{figure}[!htb]
    \centering
    \caption{Exemplo da estrutura de uma árvore KD}
    \includegraphics[width=0.5\textwidth]{./04-figuras/figura-exemplo}
    \fonte{\citeonline{Souza2012}}
    \label{fig:kdtree}
\end{figure}

\section{Quadros e tabelas}
\label{sec:tabelas}

Também é apresentado o exemplo do \autoref{qua:comparabd} e da \autoref{tab:testes}, que aparece automaticamente na lista de quadros e tabelas.

Informações sobre a construção de tabelas no \LaTeX{} podem ser encontradas na literatura especializada \cite{Lamport1986,Buerger1989,Kopka2003,Mittelbach2004}.

\begin{quadro}[!htb]
    \centering
    \caption{Hierarquia de restrições das questões.}
    \begin{tabular}{|p{7cm}|p{7cm}|}
        \hline
        \textbf{BD Relacionais}                                                                       & \textbf{BD Orientados a Objetos}                  \\
        \hline
        Os dados são passivos, ou seja, certas operações limitadas podem ser automaticamente acionadas quando os dados são usados.
        Os dados são ativos, ou seja, as solicitações fazem com que os objetos executem seus métodos. & Os processos que usam dados mudam constantemente. \\
        \hline
    \end{tabular}
    \label{qua_comparabd}
    \fonte{\citeonline{Carvalho2001}}
\end{quadro}


Muitos confundem, mas existem diferenças entre tabelas e quadros.

Um quadro é formado por linhas horizontais e verticais, sendo, portanto ``fechado''. Você deverá utilizar um quadro quando o conteúdo é majoritariamente não-numérico. O número do quadro e o título vem acima do quadro, e a fonte, deve vir abaixo.

Uma tabela é formada apenas por linhas verticais, sendo, portanto ``aberta''. Você deverá utilizar uma tabela quando o conteúdo é majoritariamente numérico. O número da tabela e o título vem acima da tabela, e a fonte, deve vir abaixo, tal como no quadro.

Exemplo de tabela:

\begin{table}[!htb]
    \centering
    \caption[Resultado dos testes]{Resultado dos testes.
    \label{tab:testes}}
    \begin{tabular}{rrrrr}
        \toprule
            & Valores 1 & Valores 2 & Valores 3 & Valores 4 \\
        \midrule
            Caso 1 & 0,86 & 0,77 & 0,81 & 163 \\
            Caso 2 & 0,19 & 0,74 & 0,25 & 180 \\
            Caso 3 & 1,00 & 1,00 & 1,00 & 170 \\
        \bottomrule
    \end{tabular}
\end{table}


\section{Equações}
\label{sec:equacoes}

A transformada de Laplace é dada na \autoref{eq:laplace}, enquanto a Eq. \ref{eq:dft} apresenta a formulação da transformada discreta de Fourier bidimensional\footnote{Deve-se reparar na formatação esteticamente perfeita destas equações.}. Observe que utilizamos propositalmente duas formas distintas para referenciar as equações.

\begin{equation}
    X(s) = \int\limits_{t = -\infty}^{\infty} x(t) \, \text{e}^{-st} \, dt
    \label{eq:laplace}
\end{equation}

\begin{equation}
    F(u, v) = \sum_{m = 0}^{M - 1} \sum_{n = 0}^{N - 1} f(m, n) \exp \left[ -j 2 \pi \left( \frac{u m}{M} + \frac{v n}{N} \right) \right]
    \label{eq:dft}
\end{equation}

\section{Algoritmos}\label{sec:algoritmos}

Os algoritmos devem ser feitos segundo o modelo abaixo. Para isso, utilizar o pacote {\ttfamily algorithm2e} no início do arquivo principal como neste exemplo.
\\
\\

\begin{algorithm}
    \caption{Algoritmo para remoção aleatória de vértices}
    \KwIn{o número $n$ de vértices a remover, grafo original $G(V, E)$}
    \KwOut{grafo reduzido $G'(V,E)$}
    $removidos \leftarrow 0$ \\
    \While {removidos $<$ n } {
        $v \leftarrow$ Random$(1, ..., k) \in V$ \\
            \For {$u \in adjacentes(v)$} {
                remove aresta (u, v)\\
                $removidos \leftarrow removidos + 1$\\
            }
            \If {há  componentes desconectados} {
                remove os componentes desconectados\\
            }
        }
\end{algorithm}


% -----------------------------------------------------------------------------
% Novo apêndice
% -----------------------------------------------------------------------------

\chapter{Sobre as listas}
\label{chap:apSobreLista}

Para construir listas de "\textit{bullets}"{} ou listas enumeradas, inclusive listas aninhadas, é utilizado o pacote \verb|paralist|.

O exemplo a seguir ilustra duas listas não numeradas aninhadas, utilizando o ambiente \verb|\itemize|. Observe a indentação, bem como a mudança automática do tipo de "\textit{bullet}"{} nas listas aninhadas.

\begin{itemize}
    \item item não numerado 1
    \item item não numerado 2
    \begin{itemize}
        \item subitem não numerado 1
        \item subitem não numerado 2
        \item subitem não numerado 3
    \end{itemize}
    \item item não numerado 3
\end{itemize}

Por outro lado, o exemplo a seguir ilustra duas listas numeradas aninhadas, utilizando o ambiente \verb|\enumerate|. Observe a numeração progressiva e indentação das listas aninhadas.

\begin{enumerate}
    \item item numerado 1
    \item item numerado 2
    \begin{enumerate}
        \item subitem numerado 1
        \item subitem numerado 2
        \item subitem numerado 3
    \end{enumerate}
    \item item numerado 3
\end{enumerate}

Cabe ressaltar que os ambientes \verb|\itemize| e \verb|\enumerate| podem ser utilizados alternativamente. No entanto, durante a compilação pdflatex são apresentados erros associados a estes ambientes, porém o pdf é gerado corretamente. Trata-se de um "\textit{bug}"{} que ainda não conseguimos resolver. Caso conheça a solução, por favor, comunique-nos para que possamos incluí-la numa futura atualização deste modelo.

% -----------------------------------------------------------------------------
% Novo apêndice
% -----------------------------------------------------------------------------

\chapter{Sobre citações e chamadas de referências}
\label{chap:apSobreCita}

Citações são trechos transcritos ou informações retiradas das publicações consultadas para a realização do trabalho.
As citações são utilizadas no texto com o propósito de esclarecer, completar, embasar ou corroborar as ideias do autor.

Todas as publicações consultadas e efetivamente utilizadas (por meio de citações) devem ser listadas, obrigatoriamente, nas referências bibliográficas, de forma a preservar os direitos autorais e intelectuais.

A norma ABNT NBR:10520-2002 classifica as citações em: citações livres e citações literais.

\section{Citações livres}
\label{sec:citacoesLivres}

Nas citações livres, reproduzem-se as ideias e informações de um autor, sem, entretanto, ``copiar letra por letra'' o texto do autor. Sendo assim, não há muito a dizer sobre como fazer citações livres, exceto que há que se tomar o devido cuidado com o "recortar e colar e modificar"{} para que não se caracterize plágio.

Quanto à chamada da referência, ela pode ser feita de duas maneiras distintas, conforme o nome do(s) autor(es) façam parte do seu texto ou não. Os exemplos a seguir ilustram estas duas possibilidades.

Enquanto \citeonline{Maturana2003} defendem uma epistemologia baseada na biologia. Para os autores, é necessário rever \ldots.

Por outro lado, \citeonline{Barbosa2004} contra-argumenta afirmando que \ldots.

Acima, as chamadas de referências foram feitas com o comando \verb|\citeonline{chave}|, que produzirá a formatação correta, conforme a norma ABNT.

Observe que em ambos os casos anteriores, a frase fica incompleta e incompreensível caso as palavras "Maturana e Varela"{} e "Barbosa et al."{} não sejam "pronunciadas"{}. Ou seja, os nomes dos autores fazem parte da frase. Neste caso, a formatação automática da chamada de referência coloca os nomes dos autores seguido, entre parêntesis pelo ano de publicação da obra referenciada. Isso apenas no caso em que se usa o esquema autor-ano, que é \textit{padrão} neste modelo \LaTeX{}.

A segunda maneira de fazer uma chamada de referência deve ser utilizada quando se quer evitar uma interrupção na sequência do texto, o que poderia, eventualmente, prejudicar a leitura.

Assim, a citação livre é feita e imediatamente após a obra referenciada deve ser colocada entre parênteses. Porém, neste caso específico, o nome do autor deve vir em caixa alta, seguido do ano da publicação, como nos exemplos a seguir.

Há defensores da epistemologia baseada na biologia que argumentam em favor da necessidade de \ldots \cite{Maturana2003}.

Por outro lado, há os que contra-argumentam afirmando que \ldots  \cite{Barbosa2004}.

Nos dois casos imediatamente acima a chamada de referência deve ser feita com o comando \verb|\cite{chave}|, que produzirá a formatação correta, conforme a norma ABNT.

Observe que o estilo de redação das frases teve que ser modificado para torná-las compreensíveis sem a menção explícita dos nomes dos autores. Estes agora não são parte integrante da frase, ficam entre parêntesis. Neste caso, a formatação automática da chamada de referência coloca, entre parêntesis, os nomes dos autores seguido pelo ano de publicação da obra referenciada. Novamente, apenas no caso em que se usa o esquema autor-ano, que é \textit{padrão} neste modelo \LaTeX{}.

Por fim, cabe chamar a atenção para o detalhe do termo \textit{et al.} que deve ser utilizado quando o trabalho citado possui mais de três autores. Esse recurso é automatizado pelo estilo {\ttfamily abntex2}. Caso não haja desejo em abreviar o nome dos demais autores através do termo \textit{et al.}, deve-se incluir a opção {\ttfamily abnt-no-etal-label}.

\section{Citações literais}
\label{sec:citacoesLiterais}

Nas citações literais, reproduzem-se as ideias e informações de um autor, exatamente como este a expressou, ou seja, faz-se uma ``cópia letra por letra'' do texto do autor. Sendo assim, obviamente, a obra citada deve ser referenciada, sob pena de se caracterizar plágio.

Quanto à chamada da referência, ela pode ser feita de qualquer das duas maneiras mencionadas na \autoref{sec:citacoesLivres}, conforme o nome do(s) autor(es) façam parte do seu texto ou não.

Há duas maneiras distintas de se fazer uma citação literal, conforme o trecho citado seja longo ou curto.

Quando o trecho citado é longo (4 ou mais linhas) deve-se usar um parágrafo específico para a citação, na forma de um texto recuado (4 cm da margem esquerda), com tamanho de letra menor do aquela utilizada no texto e espaçamento entrelinhas simples. Veja o exemplo abaixo.

\begin{citacao}
    Desse modo, opera-se uma ruptura decisiva entre a reflexividade filosófica, isto é a possibilidade do sujeito de pensar e de refletir, e a objetividade científica.     Encontramo-nos num ponto em que o conhecimento científico está sem consciência. Sem consciência moral, sem consciência reflexiva e também subjetiva. Cada vez mais o desenvolvimento extraordinário do conhecimento científico vai tornar menos praticável a própria possibilidade de reflexão do sujeito sobre a sua pesquisa \cite[p.~28]{Silva2000}.
\end{citacao}

Para se criar o efeito demonstrado na citação anterior, deve-se utilizar o comando:

\begin{verbatim}
\begin{citacao}
<citacao>
\end{citacao}
\end{verbatim}

Acima, para a chamada da referência o comando \verb|\cite[p.~28]{Silva2000}| foi utilizado, visto que os nomes dos autores não são parte do trecho citado.

Observe ainda que foi indicado o número da página da obra citada que contém o trecho citado. A localização precisa do trecho citado deve ser indicada sempre, exceto para artigos científicos (tipicamente com poucas páginas, o que geralmente não é o caso de artigos de revisão de literatura) e outros documentos com "poucas"{} páginas.

Alternativamente, é possível construir uma frase que contenha os autores, e irá encaminhar (por assim dizer) a citação literal. Assim sendo, note que pode após a citação literal não mais aparece o nome dos autores, visto que já se encontra no texto. Veja o exemplo seguinte.

No entanto, \citeonline[p.~33]{Silva2000}, ao fazerem as suas críticas à ciência moderna, afirmam:

\begin{citacao}
    Mas o curioso é que o conhecimento científico que descobriu os meios realmente extraordinários para, por exemplo, ver aquilo que se passa no nosso sol, para tentar conceber a estrutura das estrelas extremamente distantes, e até mesmo para tentar pesar o universo, o que é algo de extrema utilidade, o conhecimento científico que multiplicou seus meios de observação e de concepção do universo, dos objetos, está completamente cego, se quiser considerar-se apenas a si próprio!
\end{citacao}

Já quando o trecho citado é curto (3 ou menos linhas) ele deve inserido diretamente no texto entre aspas. Veja os dois exemplos seguintes, cada qual utilizando uma forma de chamada de referência.

A epistemologia baseada na biologia parte do princípio de que ``assumo que não posso fazer referência a entidades independentes de mim para construir meu explicar'' \cite[p.~35]{Maturana2003}.

A epistemologia baseada na biologia de \citeonline[p.~35]{Maturana2003} parte do princípio de que ``assumo que não posso fazer referência a entidades independentes de mim para construir meu explicar''.

Finalmente, e isto vale para citações curtas ou longas, caso seja necessário inserir ou suprimir (modificar de modo geral) qualquer palavra ou frase no trecho citado literalmente, qualquer que seja a finalidade, isto deve ser feito colocando sua intervenção entre colchetes retos e deve ser indicado explicitamente ao final da citação. Veja o exemplo seguinte.

A epistemologia baseada na biologia parte do princípio de que ``assumo que não posso fazer referencia [\textit{sic}] a \underline{entidades independentes} de mim [realidade objetiva] para construir meu explicar'' \cite[p.~35, comentários e grifo nosso]{Maturana2003}.

\section{Mais detalhes sobre as chamadas de referências}
\label{sec:referUtilizadas}

A seguir há mais exemplos dos comandos para as chamadas de referências e o resultado produzido.

\citeonline{Maturana2003} \ \ \  \verb|\citeonline{Maturana2003}|\\
\citeonline{Barbosa2004} \ \ \   \verb|\citeonline{Barbosa2004}|\\
\cite[p.~28]{Silva2000} \ \ \  \verb|\cite[p.~28]{Silva2000}|\\
\citeonline[p.~33]{Silva2000} \ \ \   \verb|\citeonline[p.~33]{v}|\\
\cite[p.~35]{Maturana2003} \ \ \   \verb|\cite[p.~35]{Maturana2003}|\\
\citeonline[p.~35]{Maturana2003} \ \ \   \verb|\citeonline[p.~35]{Maturana2003}|\\
\cite{Barbosa2004,Maturana2003} \ \ \   \verb|\cite{Barbosa2004,Maturana2003}|\\

Há que se tomar bastante cuidado com referências cujos autores têm nomes compostos, tipo João de Souza Júnior ou Antônio José da Silva Filho. Para que a formatação seja correta, os nomes dos autores no arquivo {\ttfamily .bib} deverá ser cadastrado de uma forma específica. Para maiores detalhes, veja o \autopageref{chap:anexoB} \footnote{O texto do anexo é de inteira responsabilidade do autor devidamente referenciado.}.

Os exemplos abaixo ilustram a formatação correta.

\cite[p.~28]{vanGELDER1998} \ \ \ \ \  \verb|\cite[p.~28]{vanGELDER1998}|\\
\citeonline[p.~28]{vanGELDER1998} \ \ \ \ \  \verb|\citeonline[p.~28]{vanGELDER1998}|\\

Observe que, a despeito do que está dito no \autoref{chap:anexoB}, ainda há falhas na formatação ABNT de nomes tipo \verb|van Gelder| quando utilizados em chamadas de referências que fazem parte do texto (\textit{e.g.}, \verb|\citeonline{vanGELDER1998}| que deveria produzir \verb|van Gelder (1998)|). Este problema pode ser contornado facilmente, simplesmente evitando o uso dessa forma de chamada de referência, preferindo sempre nestes casos, o uso da forma \textit{e.g.}, \verb|\cite{vanGELDER1998}| que será formatada corretamente, produzindo \verb|(van GELDER, 1998)|.

Observe ainda o caso em que é feita duas citações juntas \cite{Santos2003, Neubert2001} e como citar endereços Web \cite{IRL2014}.

% -----------------------------------------------------------------------------
% Novo apêndice
% -----------------------------------------------------------------------------

\chapter{Sobre as referências bibliográficas}
\label{chap:apSobreRefer}

A bibliografia é feita no padrão \textsc{Bib}\TeX{}. As referências são colocadas em um arquivo separado. Os elementos de cada item bibliográfico que devem constar nas referências bibliográficas são apresentados a seguir. Tais referências bibliográficas devem seguir a norma \citeonline{NBR6023:2002} da ABNT\footnote{As normas técnicas da ABNT não são gratuitas.}.

\section{Entradas de referências}
\label{sec:entradasRefs}

Entradas são objetos de citação bibliográficas. Dito de outra forma, são as categorias dos tipos de documentos e materiais componentes da bibliografia. A classe abn\TeX{} define as seguintes entradas:

\begin{verbatim}
@book
@inbook
@article
@phdthesis
@mastersthesis
@monography
@techreport
@manual
@proceedings
@inproceedings
@journalpart
@booklet
@patent
@unpublished
@misc
\end{verbatim}

Cada entrada é formatada pelo pacote \citeonline{abnTeX22014d} de uma forma específica. Algumas entradas foram introduzidas especificamente para atender à norma \citeonline{NBR6023:2002}, são elas: \verb|@monography|, \verb|@journalpart|,\verb|@patent|. As demais entradas são padrão \textsc{Bib}\TeX{}. Para maiores detalhes, refira-se a \citeonline{abnTeX22014d}, \citeonline{abnTeX22014b}, \citeonline{abnTeX22014c}.

A entrada \verb|@monography| é utilizada para cadastrar referências a trabalhos de conclusão de curso, monografias de cursos de especialização (pós-graduação \textit{lato sensu}), e outros trabalhos monográficos, exceto dissertação de mestrado e tese de doutorado. Eu particularmente, não considero que a formatação deste tipo de entrada está adequado. Para um trabalho de conclusão de curso (TCC) de curso de graduação, que deveria ser formatado como "[\ldots] Trabalho de Conclusão de Curso (Bacharelado em Engenharia de Computação) [\ldots]"{}; no entanto o uso de \verb|@monography| irá produzir "[\ldots] Monografia (Bacharelado em Engenharia de Computação) [\ldots]"{}. A própria  \citeonline{NBR6023:2002}, na seção 8.11.4, apresenta um exemplo com a formatação diferente daquela proporcionada por \citeonline{abnTeX22014d}.

A entrada \verb|@journalpart| é utilizada, conforme diz o manual \cite{abnTeX22014d}, para cadastrar referências e formatar partes de periódicos. Não fica claro o que se quer dizer com partes de journal. Em alguns casos, tais partes são artigos - e portanto, deveriam ser registradas como \verb|@article| - noutros casos, parece serem matérias ou textos em revistas ou jornais (não científicos). Salvo melhor juízo, me parece que esta entrada deve ser utilizada apenas neste último contexto.

A entrada \verb|@patent| é utilizada, obviamente, para cadastrar referências a patentes.

Todavia, o fato é que a normalização de referências conforme a norma \citeonline{NBR6023:2002} requer que muitos dos campos do \textsc{Bib}\TeX{} sejam adaptados. Sendo mais explícito, ao baixar um arquivo {\ttfamily .bib} de um trabalho, principalmente ser for internacional, e inserí-lo "\textit{as is}"{} em suas referências, há grande chance dessa referência ser formatada de modo errado, no que concerne à norma \citeonline{NBR6023:2002}. Isso é especialmente válido em alguns tipos de documentos de largo uso no meio acadêmico afim às áreas de Ciências Exatas, da Terra e Engenharias.

Diante disso, para evitar erros de formatação, o correto é após baixar o arquivo {\ttfamily .bib} de um trabalho, editá-lo com um editor ASCII (usando codificação UTF8), para verificar se os campos descritores que o \textit{publishers} original utilizou são aqueles requeridos pela norma ABNT.

Neste contexto, e para esta finalidade, nas seções seguintes é apresentado uma série de exemplos, quase todos, utilizados como exemplos na própria norma \citeonline{NBR6023:2002}. Para detalhes dos campos utilizados confira o arquivo {\ttfamily myRefs.bib}. Deve-se estar atento para o fato de que o uso de um sistema de gerenciamento de referências para abrir e/ou editar o arquivo {\ttfamily myRefs.bib}, pode ocultar campos utilizados pela norma ABNT e, por outro lado, exibir campos não utilizados por ela. Ou seja, o aplicativo deve ser configurado adequadamente para exibir \textbf{todos os campos}, mesmo os opcionais.

\section{Notas de rodapé}
\label{sec:notasRodape}

A norma \citeonline{NBR10520:2002}, em sua seção \textbf{7 Notas de rodapé}, classifica as notas de rodapé em duas categorias: notas explicativas\footnote{é o tipo mais comum de notas que destacam, explicam e/ou complementam o que foi dito no corpo do texto, como esta nota de rodapé, por exemplo.} e notas de referências. Já as notas de referências, como o próprio nome ja indica, são utilizadas para colocar referências e/ou chamadas de referências sob certas condições.

\subsection{Notas de referências: uso de idem, ibidem, opus citatus e outros}
\label{subsec:notasRefs}

Como indica o próprio nome, as notas de referências se prestam como recurso auxiliar para referenciação de bibliografia, e seu uso e aplicação são descritos na seção \textbf{7.1 Notas de referência} da norma \citeonline{NBR10520:2002}.

Estes recursos se referem ao uso de certas expressões consagradas para facilitar a elaboração de referencias. São eles:

\begin{itemize}
    \item idem = mesmo autor,
    \item ibidem = mesma obra,
    \item opus citatum = obra citada,
    \item locus citatum = no lugar citado,
    \item passim = aqui e alí,
    \item cf = confira,
    \item et sequentia = e sequência.
\end{itemize}

Observe que estes recursos não se adequam para serem utilizados em listas de referências bibliográficas, nem tampouco no corpo do texto. Assim, devem ser utilizados apenas nas notas de referência posicionadas no rodapé \cite[p.~6]{abnTeX22014c}, quando se referem a uma referência já feita anteriormente no corpo do texto. Ademais, essas expressões fazem sentido apenas quando aplicadas a citações de uma única referência por vez. Enfim, trata-se mais de um recurso estilístico do que algo de primeira necessidade, pelo menos para o tipo de documento usualmente elaborados nas área de Ciências Exatas, da Terra e Engenharias.

Veja o uso desses tipos de expressões nos exemplos seguintes:

\Idem[p.~21]{abnTeX22014d}

\Ibidem[p.~7]{abnTeX22014c}

\opcit[p.~9]{abnTeX22014c}

\passim{abnTeX22014c}

\cfcite[p.~3]{abnTeX22014b}

\etseq[p.~6]{abnTeX22014c}

\section{Datas em referências}
\label{sec:datasBib}

Quando as chamadas de referências são feitas no modelo autor-ano, como é o caso deste modelo \LaTeX{}, é evidente que o autor e sobretudo o ano adquirem papel de destaque. No caso da data de publicação, esta deve sempre estar presente (é elemento essencial) e indicada em algarismos arábicos.

A norma ABNT não permite o uso de expressões do tipo "sem data"{} ("[s.d]") para indicar que não se sabe a data de publicação de certa referência bibliográfica. Assim sendo, quando a data não puder ser indicada precisamente, deve-se registrar uma data aproximada entre colchetes, conforme descrito a seguir:

[1971 ou 1972] \ \ \ \ \ um ano ou outro,

[1969?] \ \ \ \ \ ano provável,

[1973] \ \ \ \ \ ano certo, não indicada no item,

[entre 1906 e 1912] \ \ \ \ \ use intervalos menores de 20 anos,

[ca. 1960] \ \ \ \ \ \textit{circa} de ... ( data aproximada),

[197-] \ \ \ \ \ década certa,

[197-?] \ \ \ \ \ década provável,

[18--] \ \ \ \ \ século certo,

[18--?] \ \ \ \ \ século provável.

\end{apendicesenv}
             % Apêndices
%
% Documento: Anexos
%

\begin{anexosenv}
\partanexos

\chapter{Nome do Anexo}
\label{chap:anexox}

Inserir seu texto aqui...

\chapter{Nome do Anexo}
\label{chap:anexoy}

Inserir seu texto aqui...

\end{anexosenv}
                % Anexos
% -----------------------------------------------------------------------------
% Índice Remissivo
% -----------------------------------------------------------------------------

% -----------------------------------------------------------------------------
% Este comando gera automaticamente o índice remissivo para os termos definidos
% no corpo do documento.
% -----------------------------------------------------------------------------

\printindex

% -----------------------------------------------------------------------------
% Não é necessário editar este arquivo.
% -----------------------------------------------------------------------------
      % Índice Remissivo

\end{document}
